\documentclass[10pt,pdf,utf8,russian,aspectratio=169,xcolor=dvipsnames,x11names,center]{beamer}

% Standard packages

\usepackage[T2A]{fontenc}
\usepackage[english,russian]{babel}
\usepackage[utf8]{inputenc}

% Setup appearance:

\usetheme{Hannover}
\usecolortheme{sidebartab}
\usecolortheme{dove}
\usefonttheme{structurebold}
\setbeamertemplate{navigation symbols}{}

%\usepackage{tgheros}
%\usepackage{tgadventor}
%\usepackage{libertine}

\usepackage{listings}

\lstdefinelanguage{javascript}{
  keywords={break, case, catch, continue, debugger, default, delete, do, else, finally, for, function, if, in, instanceof, new, return, switch, this, throw, try, typeof, var, void, while, with},
  morecomment=[l]{//},
  morecomment=[s]{/*}{*/},
  morestring=[b]',
  morestring=[b]",
  sensitive=true
}

\definecolor{light-gray}{gray}{0.95}
\lstset{
language=Python,                             % Code langugage
basicstyle=\small,                   % Code font, Examples: \footnotesize, \ttfamily
keywordstyle=\color{OliveGreen},        % Keywords font ('*' = uppercase)
commentstyle=\color{gray},              % Comments font
numbers=left,                           % Line nums position
numberstyle=\tiny,                      % Line-numbers fonts
stepnumber=1,                           % Step between two line-numbers
numbersep=5pt,                          % How far are line-numbers from code
backgroundcolor=\color{light-gray},     % Choose background color
frame=lines,                            % A frame around the code
tabsize=4,                              % Default tab size
captionpos=b,                           % Caption-position = bottom
breaklines=true,                        % Automatic line breaking?
breakatwhitespace=false,                % Automatic breaks only at whitespace?
showspaces=false,                       % Don't make spaces visible
showstringspaces=false,                 % Don't show spaces in string literals 
showtabs=false,                         % Don't make tabs visible
}

% Setup TikZ

\usepackage{tikz}
\usetikzlibrary{arrows}
\tikzstyle{block}=[draw opacity=0.7,line width=1.4cm]

% timeline

\usepackage{array}
\newcommand{\foo}{\color{OliveGreen}\makebox[0pt]{\textbullet}\hskip-0.5pt\vrule width 1pt\hspace{\labelsep}}

% Author, Title, etc.

\title[Practical Python Packaging]{\Large{Practical Python Packaging}}

\author[]{\Large{Stas Rudakou}\\\small{stas@garage22.net}}

\date{Minsk Python Meetup\\November 2014}

% The main document

\begin{document}

\begin{frame}
  \titlepage
\end{frame}

%% \begin{frame}
%%   \begin{frame}[plain]
%%     \begin{tikzpicture}[remember picture,overlay]
%%       \node[at=(current page.center)] {
%%         \includegraphics[width=\paperwidth]{}
%%       };
%%     \end{tikzpicture}
%%   \end{frame}
%% \end{frame}

%%%%%%%%
\section{История}

\begin{frame}
  \begin{center}
    {\Huge История}
  \end{center}
\end{frame}

\begin{frame}
  \frametitle{История}
  \begin{table}
    \begin{tabular}{@{\,}r <{\hskip 2pt} !{\foo} >{\raggedright\arraybackslash}p{9cm}}
      2000 & {\tt distutils} вошел в стандартную библиотеку Python 1.6 \pause\\
      2003 & запущен {\tt PyPI}\pause\\
      2004 & выпуск {\tt setuptools} ({\tt easy\_install}, формат {\tt egg})\pause\\
      2006 & выпуск {\tt buildout}\pause\\
      2007 & выпуск {\tt virtualenv}\pause\\
      2008 & выпуск {\tt pip}\pause\\
      2008 & {\tt distribute} --- форк {\tt setuptools}\pause\\
      2012 & distutils2/packaging не вошел в Python 3.3\pause\\
      2012 & {\tt venv} включен в стандартную библиотеку Python 3.3\pause\\
      2012 & дистрибутив Anaconda, пакетный менеджер {\tt conda}\pause\\
      2013 & объединение проектов {\tt distribute} и {\tt setuptools}\pause\\
      2013 & {\tt pip} и {\tt easy\_install} начинают использовать SSL\pause\\
      2013 & формат {\tt Wheel}\pause\\
      2014 & в Python 3.4 вошли скрипты для установки {\tt pip}\\
    \end{tabular}
  \end{table}
\end{frame}

%%%%%%%%
\section{Пакетирование}

\begin{frame}
  \begin{center}
    {\Huge Пакетирование}
  \end{center}
\end{frame}

\begin{frame}
  \frametitle{Пакетирование}
  \centering
  
  \begin{minipage}{9cm}
    Как Dev я хочу:
    \begin{itemize}
    \item думать, как писать программы и библиотеки
    \item не думать, как распространять программы
    \item использовать чужие наработки
    \item делиться своими наработками
    \end{itemize}
  \end{minipage}

  \vspace{0.6cm}
  \pause
  \begin{minipage}{9cm}
    Как Ops я хочу:
    \begin{itemize}
    \item легко ставить и сносить программы\\(при помощи pip)
    \item иметь свой PyPI-совместимый сервер
    \end{itemize}
  \end{minipage}
\end{frame}

\subsection{setuptools}

\begin{frame}[fragile]
  \frametitle{setuptools}

  \begin{lstlisting}[escapeinside=||]
$ find .
.
./hello
./hello/data.json
./hello/__init__.py
./hello/world.py |\pause|
./tests/__init__.py
./tests/test_hello.py |\pause|

setup.py
  \end{lstlisting}
\end{frame}

\begin{frame}[fragile]
  \frametitle{setuptools}

  \begin{lstlisting}
from setuptools import setup, find_packages
setup(
    name = "hello-world",
    version = "0.1",
    packages = find_packages(exclude=["tests"]),

    install_requires = ["docutils>=0.3"],
    package_data = {
        # If any package contains *.rst files, include them
        "": ["*.rst"],
        # And include any *.json files found in the "hello" package
        "hello": ["*.json"],
    },

    author = "Stas Rudakou",
    author_email = "stas@garage22.net",
    description = "This is an Example Package",
)
  \end{lstlisting}
\end{frame}

\begin{frame}[fragile]
  \frametitle{setuptools}

  \begin{lstlisting}[escapeinside=||]
$ python setup.py install |\pause|
$ python setup.py sdist
$ ls dist
hello-world-0.1.tar.gz
  \end{lstlisting}
    
\end{frame}

\begin{frame}[fragile]
  \frametitle{setuptools}

  \begin{lstlisting}
$ find .
.
./hello
./hello/data.json
./hello/__init__.py
./hello/world.py
./tests/__init__.py
./tests/test_hello.py

setup.py
setup.cfg
README.rst
MANIFEST.in
  \end{lstlisting}
\end{frame}

\begin{frame}[fragile]
  \frametitle{setuptools}

  \begin{lstlisting}
import os.path
from setuptools import setup, find_packages

this_dir = os.path.dirname(__file__)
with open(os.path.join(this_dir, 'README.rst')) as f:
    README = f.read()

setup(
    name = "hello-world",
    version = "0.1",
    # ...
    long_description=README
)
  \end{lstlisting}
\end{frame}

\subsection{pbr}

\begin{frame}[fragile]
  \frametitle{pbr}

  \begin{itemize}
  \item {\tt pbr} (Python Build Reasonableness) --- это \ldots
  \item библиотека от разработчиков OpenStack \ldots
  \item для управления {\tt setuptools} \ldots\pause
  \item с декларативной конфигурацией \ldots
  \item и версиями на основе ревизий и тегов в git\pause
  \end{itemize}

  \pause

  \begin{lstlisting}
from setuptools import setup

setup(
    setup_requires=['pbr'],
    pbr=True,
)
  \end{lstlisting}

\end{frame}


\begin{frame}[fragile]
  \frametitle{pbr}

  \begin{lstlisting}[caption=setup.cfg]
[metadata]
name = hello-world
author = Stas Rudakou
author-email = stas@garage22.net
description-file = README.rst
[files]
packages =
    hello
  \end{lstlisting}

  \begin{lstlisting}[caption=requirements.txt]
docutils>=0.3
  \end{lstlisting}

\end{frame}

  
\subsection{Разработка}

\begin{frame}
  \frametitle{Разработка}

  \begin{itemize}
  \item Обновили код --- запустили {\tt python setup.py install} \ldots точно? \pause
  \item {\tt pip install -e .} \pause
  \item {\tt tox} для запуска тестов в изолированных virtualenv'ах \pause
  \item свой {\tt PyPI}-сервер --- это просто ({\tt nginx} или {\tt devpi})
  \end{itemize}
\end{frame}

\section{Приятные мелочи}

\begin{frame}
  \begin{center}
    {\Huge Приятные мелочи}
  \end{center}
\end{frame}

\subsection{pkg\_resources}

\begin{frame}[fragile]

  \frametitle{pkg\_resources}
  \centering

  \begin{itemize}
  \item модуль из состава {\tt setuptools}
  \item изначально создавался как слой доступа к ресурсам пакета\\
    (не забываем, что формат {\tt egg} появился именно в {\tt setuptools})
  \item позволяет получить метаданные любого установленного пакета
  \end{itemize}
  \pause

  \begin{lstlisting}
>>> import pkg_resources
>>> dist = pkg_resources.get_distribution("pytest")
>>> dist.project_name
"pytest"
>>> dist.version
"2.6.4"
  \end{lstlisting}

\end{frame}

\subsection{Entry points}
\begin{frame}[fragile]
  \frametitle{Entry points}
  \centering

  \begin{lstlisting}
setup(
    name="my-package",
    # other arguments here...
    entry_points={
        "group1": [
            "name1 = my_package.some_module:func1",
            "name2 = my_package.other_module",
        ],
    }
)
  \end{lstlisting}

  \begin{lstlisting}
>>> import pkg_resources
>>> for ep in pkg_resources.iter_entry_points("group1"):
...   print(ep.dist.project_name, ep.name, repr(ep.load))
my-package name1 "<function func1 at 0x7ff5091e5730>"
my-package name2 "<module 'other_module' from ... >"
other-package name3 "<function func3 at 0x7ff5091e5730>"

  \end{lstlisting}

\end{frame}

\subsection{Console scripts}
\begin{frame}[fragile]
  \frametitle{Console scripts}
  \begin{minipage}{13.7cm}

  \begin{lstlisting}
setup(
    name="mysite",
    # other arguments here...
    install_requires = ["Django>=1.7,<1.8"],
    entry_points={
        "console_scripts": [
            "mysite_manage = mysite_manage:main",
        ],
    }
)
  \end{lstlisting}

  \pause

  \begin{lstlisting}
(myvenv) $ pip install mysite
(myvenv) $ which mysite_manage
/tmp/myvenv/bin/mysite_manage
(myvenv) $ ls -l `which mysite_manage`
-rwxr-xr-x 1 stas stas 300 Nov 28 19:00 /tmp/myvenv/bin/mysite_manage
  \end{lstlisting}

  \end{minipage}
\end{frame}

\begin{frame}[fragile]
  \frametitle{Console scripts \& Django}

  \begin{lstlisting}[caption=manage.py из состава Django,basicstyle=\ttfamily\footnotesize]
import os, sys

if __name__ == "__main__":
    os.environ.setdefault("DJANGO_SETTINGS_MODULE", "mysite.settings")
    from django.core.management import execute_from_command_line
    execute_from_command_line(sys.argv)
  \end{lstlisting}

  \begin{lstlisting}[caption=mysite\_manage.py,basicstyle=\ttfamily\footnotesize]
import os,sys

def main():
    os.environ.setdefault("DJANGO_SETTINGS_MODULE", "mysite.settings")
    from django.core.management import execute_from_command_line
    execute_from_command_line(sys.argv)

if __name__ == "__main__":
    main()
  \end{lstlisting}

\end{frame}

\subsection{Plug-ins}

\begin{frame}[fragile]
  \frametitle{Plug-ins на примере pytest}

  \begin{lstlisting}[caption=setup.py pytest-плагина pytest-cov]
setup(
    name="pytest-cov",
    description="py.test plugin for coverage reporting with "
    "support for both centralised and distributed testing, "
    "including subprocesses and multiprocessing",
    # other arguments here...
    install_requires=[
        "py>=1.4.22",
        "pytest>=2.6.0",
        "coverage>=3.7.1",
        "cov-core>=1.14.0"
    ],
    entry_points={
        "pytest11": ["pytest_cov = pytest_cov"]
    },
)
  \end{lstlisting}
\end{frame}

\subsection{Extras}

\begin{frame}[fragile]
  \frametitle{Extras}

  \begin{lstlisting}
setup(
    name="celery",
    description="Distributed Task Queue",
    # other arguments here...
    extras_require={
      "cassandra": ["pycassa"],
      "mongodb": ["pymongo>=2.6.2"],
      "msgpack": ["msgpack-python>=0.3.0"],
      "redis": ["redis>=2.8.0"],
      "zeromq": ["pyzmq>=13.1.0"],
    }
)
  \end{lstlisting}
  \pause
  \vspace{0.5cm}
  \begin{lstlisting}[numbers=none]
$ pip install "celery[mongodb,msgpack,zeromq]"
  \end{lstlisting}
\end{frame}

\section{Проблемы}

\begin{frame}
  \begin{center}
    {\Huge Проблемы}
  \end{center}
\end{frame}

\begin{frame}[fragile]
  \frametitle{Проблемы и ограничения}

  \begin{itemize}
  \item инструменты не всегда следят за конфликтами
  \item нет понятия ``поставлен по зависимости''
  \item неочевидные различия между
    \begin{lstlisting}[numbers=none]
pip install X Y
    \end{lstlisting}
    \begin{lstlisting}[numbers=none]
pip install X
pip install Y
    \end{lstlisting}
    \begin{lstlisting}[numbers=none]
pip install Y
pip install X
    \end{lstlisting}
  \item pip ведет себя не так, как apt-get, yum и др. пакетные менеджеры
  \end{itemize}

\end{frame}

\section{Собираем все воедино}

\begin{frame}
  \frametitle{Собираем все воедино}

  \begin{enumerate}
  \item наше настоящее: {\tt virtualenv + pip + setuptools + wheel}
  \item на систему пакетирования мы хотим возложить
    \begin{itemize}
    \item кроссплатформенные решения проблем сборки и запуска
    \item типичные задачи по разбиению программы на компоненты
    \end{itemize}
  \item на систему пакетирования мы не хотим возлагать управление системой\\
    (для этого есть {\tt ansible}, {\tt chef}, {\tt puppet}, {\tt salt})
  \end{enumerate}
\end{frame}

\begin{frame}
    \begin{block}{Стас Рудаков}
    \par \url{mailto:stas@garage22.net}
    \par \url{https://raw.github.com/nott/talks/ppp.pdf}
    \end{block}

    \begin{block}
      \par \url{https://packaging.python.org/}
      \par \url{https://pythonhosted.org/setuptools/}
    \end{block}
    
\end{frame}

\end{document}
